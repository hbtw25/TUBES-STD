\documentclass[a4paper,12pt]{article}
\usepackage{graphicx}
\usepackage{geometry}
\usepackage{enumitem}
\usepackage{listings}
\usepackage{xcolor}

% Configuration for code listings
\definecolor{codegreen}{rgb}{0,0.6,0}
\definecolor{codegray}{rgb}{0.5,0.5,0.5}
\definecolor{codepurple}{rgb}{0.58,0,0.82}
\definecolor{backcolour}{rgb}{0.95,0.95,0.92}

\lstdefinestyle{mystyle}{
    backgroundcolor=\color{backcolour},   
    commentstyle=\color{codegreen},
    keywordstyle=\color{magenta},
    numberstyle=\tiny\color{codegray},
    stringstyle=\color{codepurple},
    basicstyle=\ttfamily\footnotesize,
    breakatwhitespace=false,         
    breaklines=true,                 
    captionpos=b,                    
    keepspaces=true,                 
    numbers=left,                    
    numbersep=5pt,                  
    showspaces=false,                
    showstringspaces=false,
    showtabs=false,                  
    tabsize=2
}

\lstset{style=mystyle}

% Margin configurations
\geometry{
    a4paper,
    left=3cm,
    right=3cm,
    top=3cm,
    bottom=3cm
}

\begin{document}

\begin{center}
    \textbf{\Large DESKRIPSI TUGAS BESAR}
\end{center}

\vspace{1cm}

\begin{enumerate}[label=\textbf{\arabic*.}]
    
    \item \textbf{JUDUL} : Manajemen Pembelian Senjata pada Buy Phase Game Valorant

    \item \textbf{TIPE MLL} : Tipe B

    \item \textbf{Jenis List Parent} : Single Linked List

    \item \textbf{Jenis List Child} : Doubly Linked List

    \item \textbf{Model MLL} :
    
    \begin{center}
        \fcolorbox{black}{white}{\parbox{10cm}{\centering \vspace{1cm} [GAMBAR MODEL MLL TIPE B] \\ \vspace{0.5cm} \small{(Parent SLL, Child DLL, Relasi SLL)} \vspace{1cm}}}
    \end{center}
    
    \item \textbf{Data Player (Parent)} :
    \begin{itemize}
        \item ID Player
        \item Nama Player
        \item Role (Duelist/Controller/Initiator/Sentinel)
        \item Credit (Uang)
    \end{itemize}

    \item \textbf{Data Weapon (Child)} :
    \begin{itemize}
        \item ID Weapon
        \item Nama Weapon
        \item Kategori
        \item Harga
    \end{itemize}

    \item \textbf{Spesifikasi Program (Bobot Total 100)}
    
    \begin{enumerate}[label=\alph*.]
        \item \textbf{Insert element parent}
        \begin{itemize}
            \item PIC: Harsya
            \item Status: \textbf{Done} (Function: \texttt{insertLastParent})
        \end{itemize}
\begin{lstlisting}[language=C++]
void insertLastParent(ListParent &LP, addressParent N) {
    if (LP.first == nullptr) {
        LP.first = N;
    } else {
        addressParent P = LP.first;
        while (P->next != nullptr) {
            P = P->next;
        }
        P->next = N;
    }
}
\end{lstlisting}

        \item \textbf{Insert element child}
        \begin{itemize}
            \item PIC: Raditya
            \item Status: \textbf{Done} (Function: \texttt{insertLastChild})
        \end{itemize}
\begin{lstlisting}[language=C++]
void insertLastChild(ListChild &LC, addressChild N) {
    if (LC.first == nullptr) { 
        LC.first = N; 
        LC.last = N; 
    } else {
        N->prev = LC.last;
        LC.last->next = N;
        LC.last = N;
    }
}
\end{lstlisting}

        \item \textbf{Insert element relation}
        \begin{itemize}
            \item PIC: Bersama
            \item Status: \textbf{Done} (Function: \texttt{insertFirstRelasi})
        \end{itemize}
\begin{lstlisting}[language=C++]
void insertFirstRelasi(ListRelasi &LR, addressRelasi N) {
    N->next = LR.first;
    LR.first = N;
}
\end{lstlisting}

        \item \textbf{Delete element parent}
        \begin{itemize}
            \item PIC: Harsya
            \item Status: \textbf{Done} (Function: \texttt{deleteParent} \& \texttt{dealokasiParent})
        \end{itemize}
\begin{lstlisting}[language=C++]
bool deleteParent(ListParent &LP, ListRelasi &LR, string idPlayer) {
    if (LP.first == nullptr) return false;

    addressParent prevNode = nullptr;
    addressParent cur = LP.first;

    while (cur != nullptr) {
        if (cur->info.idPlayer == idPlayer) {
            deleteRelasiByPlayerPtr(LR, cur); // Delete cascading

            if (prevNode == nullptr) LP.first = cur->next;
            else prevNode->next = cur->next;

            dealokasiParent(cur);
            return true;
        }
        prevNode = cur;
        cur = cur->next;
    }
    return false;
}
\end{lstlisting}

        \item \textbf{Delete element child}
        \begin{itemize}
            \item PIC: Raditya
            \item Status: \textbf{Done} (Function: \texttt{deleteChild} \& \texttt{dealokasiChild})
        \end{itemize}
\begin{lstlisting}[language=C++]
bool deleteChild(ListChild &LC, ListRelasi &LR, string idWeapon) {
    if (LC.first == nullptr) return false;

    addressChild cur = LC.first;
    while (cur != nullptr) {
        if (cur->info.idWeapon == idWeapon) {
            deleteRelasiByWeaponPtr(LR, cur); // Delete cascading

            if (cur->prev != nullptr) cur->prev->next = cur->next;
            else LC.first = cur->next;

            if (cur->next != nullptr) cur->next->prev = cur->prev;
            else LC.last = cur->prev;

            dealokasiChild(cur);
            return true;
        }
        cur = cur->next;
    }
    return false;
}
\end{lstlisting}

        \item \textbf{Delete element relation}
        \begin{itemize}
            \item PIC: Bersama
            \item Status: \textbf{Done} (Function: \texttt{deleteRelasi} \& \texttt{dealokasiRelasi})
        \end{itemize}
\begin{lstlisting}[language=C++]
bool deleteRelasi(ListRelasi &LR, string idPlayer, string idWeapon, int round) {
    if (LR.first == nullptr) return false;

    addressRelasi prevNode = nullptr;
    addressRelasi cur = LR.first;

    while (cur != nullptr) {
        bool match = (cur->info.pPlayer->info.idPlayer == idPlayer &&
                      cur->info.pWeapon->info.idWeapon == idWeapon &&
                      cur->info.round == round);
        if (match) {
            if (prevNode == nullptr) LR.first = cur->next;
            else prevNode->next = cur->next;

            dealokasiRelasi(cur);
            return true;
        }
        prevNode = cur;
        cur = cur->next;
    }
    return false;
}
\end{lstlisting}

        \item \textbf{Find element Parent}
        \begin{itemize}
            \item PIC: Harsya
            \item Status: \textbf{Done} (Function: \texttt{findParent})
        \end{itemize}
\begin{lstlisting}[language=C++]
addressParent findParent(ListParent LP, string idPlayer) {
    addressParent P = LP.first;
    while (P != nullptr) {
        if (P->info.idPlayer == idPlayer) return P;
        P = P->next;
    }
    return nullptr;
}
\end{lstlisting}

        \item \textbf{Find element child}
        \begin{itemize}
            \item PIC: Raditya
            \item Status: \textbf{Done} (Function: \texttt{findChild})
        \end{itemize}
\begin{lstlisting}[language=C++]
addressChild findChild(ListChild LC, string idWeapon) {
    addressChild P = LC.first;
    while (P != nullptr) {
        if (P->info.idWeapon == idWeapon) return P;
        P = P->next;
    }
    return nullptr;
}
\end{lstlisting}

        \item \textbf{Find apakah parent dan child tertentu memiliki relasi}
        \begin{itemize}
            \item PIC: Bersama
            \item Status: \textbf{Done} (Function: \texttt{cekRelasi})
        \end{itemize}
\begin{lstlisting}[language=C++]
bool cekRelasi(ListRelasi LR, string idPlayer, string idWeapon) {
    addressRelasi R = LR.first;
    while (R != nullptr) {
        if (R->info.pPlayer->info.idPlayer == idPlayer &&
            R->info.pWeapon->info.idWeapon == idWeapon) return true;
        R = R->next;
    }
    return false;
}
\end{lstlisting}

        \item \textbf{Show all data di List Parent}
        \begin{itemize}
            \item PIC: Harsya
            \item Status: \textbf{Done} (Function: \texttt{showParent})
        \end{itemize}
\begin{lstlisting}[language=C++]
void showParent(ListParent LP) {
    if (LP.first == nullptr) { cout << "Belum ada player.\n"; return; }
    addressParent P = LP.first;
    while (P != nullptr) {
        cout << P->info.idPlayer << " | " << P->info.nama << "\n";
        P = P->next;
    }
}
\end{lstlisting}

        \item \textbf{Show all data di List Child}
        \begin{itemize}
            \item PIC: Raditya
            \item Status: \textbf{Done} (Function: \texttt{showChild})
        \end{itemize}
\begin{lstlisting}[language=C++]
void showChild(ListChild LC) {
    if (LC.first == nullptr) { cout << "Belum ada weapon.\n"; return; }
    addressChild P = LC.first;
    while (P != nullptr) {
        cout << P->info.idWeapon << " | " << P->info.nama << "\n";
        P = P->next;
    }
}
\end{lstlisting}

        \item \textbf{Show data child dari parent tertentu}
        \begin{itemize}
            \item PIC: Harsya
            \item Status: \textbf{Done} (Function: \texttt{tampilWeaponPerPlayer})
        \end{itemize}
\begin{lstlisting}[language=C++]
void tampilWeaponPerPlayer(ListRelasi LR, string idPlayer) {
    addressRelasi R = LR.first;
    while (R != nullptr) {
        if (R->info.pPlayer->info.idPlayer == idPlayer) {
            cout << "- " << R->info.pWeapon->info.nama << "\n";
        }
        R = R->next;
    }
}
\end{lstlisting}

        \item \textbf{Show data parent dari child tertentu}
        \begin{itemize}
            \item PIC: Raditya
            \item Status: \textbf{Done} (Function: \texttt{tampilPlayerPerWeapon})
        \end{itemize}
\begin{lstlisting}[language=C++]
void tampilPlayerPerWeapon(ListRelasi LR, string idWeapon) {
    addressRelasi R = LR.first;
    while (R != nullptr) {
        if (R->info.pWeapon->info.idWeapon == idWeapon) {
            cout << "- " << R->info.pPlayer->info.nama << "\n";
        }
        R = R->next;
    }
}
\end{lstlisting}

        \item \textbf{Show setiap data parent beserta data child yang berelasi dengannya}
        \begin{itemize}
            \item PIC: Bersama
            \item Status: \textbf{Done} (Function: \texttt{tampilSemuaPlayerDanWeapon})
        \end{itemize}
\begin{lstlisting}[language=C++]
void tampilSemuaPlayerDanWeapon(ListParent LP, ListRelasi LR) {
    addressParent P = LP.first;
    while (P != nullptr) {
        cout << P->info.nama << ":\n";
        addressRelasi R = LR.first;
        while (R != nullptr) {
            if (R->info.pPlayer == P) {
                cout << "  - " << R->info.pWeapon->info.nama << "\n";
            }
            R = R->next;
        }
        P = P->next;
    }
}
\end{lstlisting}

        \item \textbf{Show setiap data child beserta data parent yang berelasi dengannya}
        \begin{itemize}
            \item PIC: Bersama
            \item Status: \textbf{Done} (Function: \texttt{tampilSemuaWeaponDanPlayer})
        \end{itemize}
\begin{lstlisting}[language=C++]
void tampilSemuaWeaponDanPlayer(ListChild LC, ListRelasi LR) {
    addressChild W = LC.first;
    while (W != nullptr) {
        cout << W->info.nama << ":\n";
        addressRelasi R = LR.first;
        while (R != nullptr) {
            if (R->info.pWeapon == W) {
                cout << "  - " << R->info.pPlayer->info.nama << "\n";
            }
            R = R->next;
        }
        W = W->next;
    }
}
\end{lstlisting}

        \item \textbf{Count jumlah child element parent tertentu}
        \begin{itemize}
            \item PIC: Harsya
            \item Status: \textbf{Done} (Function: \texttt{hitungTotalWeaponPlayer})
        \end{itemize}
\begin{lstlisting}[language=C++]
int hitungTotalWeaponPlayer(ListRelasi LR, string idPlayer) {
    int cnt = 0;
    addressRelasi R = LR.first;
    while (R != nullptr) {
        if (R->info.pPlayer->info.idPlayer == idPlayer) cnt++;
        R = R->next;
    }
    return cnt;
}
\end{lstlisting}

        \item \textbf{Count jumlah parent yang dimiliki oleh child tertentu}
        \begin{itemize}
            \item PIC: Raditya
            \item Status: \textbf{Done} (Function: \texttt{hitungTotalPlayerWeapon})
        \end{itemize}
\begin{lstlisting}[language=C++]
int hitungTotalPlayerWeapon(ListRelasi LR, string idWeapon) {
    int cnt = 0;
    addressRelasi R = LR.first;
    while (R != nullptr) {
        if (R->info.pWeapon->info.idWeapon == idWeapon) cnt++;
        R = R->next;
    }
    return cnt;
}
\end{lstlisting}

        \item \textbf{Count element child yang tidak memiliki parent}
        \begin{itemize}
            \item PIC: Raditya
            \item Status: \textbf{Done} (Function: \texttt{hitungWeaponTidakTerbeli})
        \end{itemize}
\begin{lstlisting}[language=C++]
int hitungWeaponTidakTerbeli(ListChild LC, ListRelasi LR) {
    int cnt = 0;
    addressChild W = LC.first;
    while (W != nullptr) {
        bool pernah = false;
        addressRelasi R = LR.first;
        while (R != nullptr) {
            if (R->info.pWeapon == W) { pernah = true; break; }
            R = R->next;
        }
        if (!pernah) cnt++;
        W = W->next;
    }
    return cnt;
}
\end{lstlisting}

        \item \textbf{Count element parent yang tidak memiliki child}
        \begin{itemize}
            \item PIC: Harsya
            \item Status: \textbf{Done} (Function: \texttt{hitungPlayerTidakMembeli})
        \end{itemize}
\begin{lstlisting}[language=C++]
int hitungPlayerTidakMembeli(ListParent LP, ListRelasi LR) {
    int cnt = 0;
    addressParent P = LP.first;
    while (P != nullptr) {
        bool pernah = false;
        addressRelasi R = LR.first;
        while (R != nullptr) {
            if (R->info.pPlayer == P) { pernah = true; break; }
            R = R->next;
        }
        if (!pernah) cnt++;
        P = P->next;
    }
    return cnt;
}
\end{lstlisting}

        \item \textbf{Edit relasi (Change Child/Parent)}
        \begin{itemize}
            \item PIC: Bersama
            \item Status: \textbf{Done} (Function: \texttt{editRelasiGantiWeapon} \& \texttt{editRelasiGantiPlayer})
        \end{itemize}
\begin{lstlisting}[language=C++]
bool editRelasiGantiWeapon(...) {
    // Cari transaksi lama, ubah pWeapon ke weapon baru
    // ...
}

bool editRelasiGantiPlayer(...) {
    // Cari transaksi lama, ubah pPlayer ke player baru
    // ...
}
\end{lstlisting}
    \end{enumerate}

    \item \textbf{Persen kontribusi anggota dalam tim (total 100\%)}

    \begin{itemize}
        \item \textbf{Harsya Brahmantyo Wibowo (103032430021) : 50\%}
        \item \textbf{Raditya Vihandika Bari Jabran (103032400011) : 50\%}
    \end{itemize}

    \item \textbf{Bukti responsi tugas besar bersama asdos, asprak, ataupun dosen.}
    
    \begin{center}
        \vspace{2cm}
        \fcolorbox{black}{white}{\parbox{10cm}{\centering \vspace{2cm} [TEMPEL FOTO BUKTI RESPONSI DISINI] \vspace{2cm}}}
    \end{center}

\end{enumerate}

\end{document}
